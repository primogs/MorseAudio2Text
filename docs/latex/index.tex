\hypertarget{index_Introduction}{}\section{Introduction}\label{index_Introduction}
Sventythree converts a morse signal from a audio stream to a text. The audio stream is recorded with the sound card. A recorded audio file can be decoded if the soundcard output is conected with the input by a wire. An additional prefilter is maybe required if the audio stream is very noisy. For more information about the decoding see \hyperlink{classMorseDecode}{Morse\+Decode} class.\hypertarget{index_Configuration}{}\section{File}\label{index_Configuration}
A configuration file can be used to parameterize the programm. The configuration must be named \char`\"{}config.\+txt\char`\"{} and placed in the same folder as the executable file. Following parameter are supported\+: Max\+Memory\+Consumption, Audio\+In\+Sample\+Rate, Max\+Amplitude, Auto\+Threshold\+Factor, Min\+Threshold, Dot\+Time\+Lower\+Limit, Dot\+Time\+Upper\+Limit, Stable\+Dot\+Time\+Inaccuracy, Max\+Morse\+Signs\+Per\+Char, Debounce\+Bounce\+Time, Low\+Pass\+Decay\+Rate, Auto\+Threshold\+Decay\+Rate, Ave\+Time\+Buffer\+Length, Short\+Time\+Buffer\+Length,Stable\+Dot\+Time\+Buffer\+Length, Edge\+Event\+Buffer\+Length, Text\+Buffer\+Length A parameter is expected as name value pair in a single row eg. Audio\+In\+Sample\+Rate 44100 The value must be in the base unit, eg. seconds if it is a time. Any line starting with a \# will be ignorde.\hypertarget{index_Source}{}\section{Code}\label{index_Source}
Sventythree use wx\+Widgets (version 3.\+0) for the gui and portaudio (v19) for audio card access. You will need to link against both libaries to get the full functionality. It is also possible to compile the programm as command line tool reading from stdin. 